% This is samplepaper.tex, a sample chapter demonstrating the
% LLNCS macro package for Springer Computer Science proceedings;
% Version 2.21 of 2022/01/12
%
\documentclass[runningheads]{llncs}
%
\usepackage[T1]{fontenc}
% T1 fonts will be used to generate the final print and online PDFs,
% so please use T1 fonts in your manuscript whenever possible.
% Other font encondings may result in incorrect characters.
%
\usepackage{graphicx}
% Used for displaying a sample figure. If possible, figure files should
% be included in EPS format.
%
% If you use the hyperref package, please uncomment the following two lines
% to display URLs in blue roman font according to Springer's eBook style:
%\usepackage{color}
%\renewcommand\UrlFont{\color{blue}\rmfamily}
%\urlstyle{rm}
\graphicspath{{./figures}}
\usepackage{url}
\usepackage{comment}
\usepackage{multirow}
\usepackage{subcaption}
\usepackage{caption}
\usepackage{float}
%\usepackage{refcheck}
\begin{document}
%
\title{Multi-Floor IPS, A Simplified Indoor Positioning System Study}
%
%\titlerunning{Abbreviated paper title}
% If the paper title is too long for the running head, you can set
% an abbreviated paper title here
%
\author{First Author\inst{1}\orcidID{0000-1111-2222-3333} \and
Second Author\inst{2,3}\orcidID{1111-2222-3333-4444} \and
Third Author\inst{3}\orcidID{2222--3333-4444-5555}}
%
\authorrunning{F. Author et al.}
% First names are abbreviated in the running head.
% If there are more than two authors, 'et al.' is used.
%
\institute{Princeton University, Princeton NJ 08544, USA \and
Springer Heidelberg, Tiergartenstr. 17, 69121 Heidelberg, Germany
\email{lncs@springer.com}\\
\url{http://www.springer.com/gp/computer-science/lncs} \and
ABC Institute, Rupert-Karls-University Heidelberg, Heidelberg, Germany\\
\email{\{abc,lncs\}@uni-heidelberg.de}}
%
\maketitle              % typeset the header of the contribution
%
\begin{abstract}
Indoor positioning systems are a relatively new positioning tool that aims to supplement or replace the usage of Global Positioning Tools for indoor positioning. However; there are minimal studies on Indoor positioning systems in a multi-floor model. This paper aims to provide further knowledge on multi-floor indoor positioning models through demonstrating and analyzing the process of creating a multi-floor indoor positioning model, analysis of machine learning models and how well they perform in a multi-floor indoor positioning model. This paper also aims to introduce two new metrics, Average Grid from Target and Average Distance from Target to better quantify IPS performance as well as the effects of feature filtering, grid size, and data point density on positioning accuracy. The paper also aims to use Average Grid from Target (AGT) to find the optimal grid-size for the location by analysing varying grid sizes on the machine learning model. The experiment was conducted on the 6th and 7th floor hallways of an university learning center. With each part of the hallway being segmented into 1x1 metre grids. Two Android devices running a modified open-source IPS data collection application were used to gather RSSI data across approximately 600 grids per floor. A dataset of 12,640 points was collected across two floors using a filtered set of 378 BSSIDs containing specific identifiers Experiments with different grid sizes showed that different parameter settings are better for optimizing solely for accuracy or for Average Grid from Target (AGT). The study concludes that a 7×7m grid size offers the best balance between accuracy and precision for this specific university learning center environment. A simple WiFi access points filter was implemented. It lowered training time, computational load as well as slightly improved model stability. 

\keywords{First keyword  \and Second keyword \and Another keyword.}
\end{abstract}
%
%
%
\newpage
\section{Introduction}
Indoor Positioning Systems (IPS) address navigation challenges in enclosed environments where GPS accuracy degrades significantly. Dense building materials cause signal scattering, shadowing, and attenuation, rendering GPS unreliable indoors \cite{bgp1}. This limitation has driven the development of alternative IPS approaches.

Received Signal Strength Indicator (RSSI) fingerprinting represents a widely adopted IPS approach, utilizing signal strength measurements from transmitters like Bluetooth Low Energy (BLE) devices \cite{bg2}. The technique involves offline radio map creation followed by online position estimation through triangulation. However, RSSI measurements suffer from environmental noise and require precise access point placement, limiting accuracy \cite{bgp2}.

Machine learning approaches have proven effective for IPS performance improvement \cite{bgp3}. Classification algorithms including Support Vector Machine (SVM), k-Nearest Neighbor (kNN), Random Forests, and Neural Networks address RSSI limitations by treating positioning as a classification problem, providing reliable area-based location estimates.

Previous research implemented an IPS using a large grid size (16.75$\times$15m) to reduce the number of classification labels, making the problem more manageable within time constraints. While this approach provided a functional implementation, it left several questions unanswered regarding data point influence on IPS performance, the feasibility of implementing reliable IPS with limited Basic Service Set Identifier (BSSID) features, and the minimum viable grid size for maintaining positioning accuracy.
Building on this foundation, this paper further explores classification-based IPS by refining the approach to ground truth reliability in experimental settings. The main contributions of this paper are as follows:
\begin{enumerate}
	\item Introduction of two new evaluation metrics: Average Grid from Target (AGT) and Average Distance from Target~(ADT).
	\item Investigation of trade-offs between grid size and precision, analyzing implementation advantages and limitations.
	\item Presentation of deeper insights into IPS design considerations, offering practical improvements for indoor positioning accuracy.
	\item Examination of feature filtering effects on model complexity.
\end{enumerate}

%%%%%%%%%%%%%%%%%%%%%%%%%%%%%%%%%%%%%%%%%%%%%%%%%%%%%%%%%%%%%%%%%%%%%%%%%%%%%%%%

\section{Literature Review}
GPS limitations indoors arise from signal attenuation through dense building materials, resulting in scattering and positioning errors \cite{bgp1}. This necessitates alternative indoor positioning approaches.

IPS have been extensively studied using Wi-Fi RSSI fingerprinting enhanced through accessible ML algorithms. Traditional algorithms including k-Nearest Neighbor (kNN) \cite{LRE1}, \cite{LRE2}, \cite{LRE6}, Random Forest (RF) \cite{LRE1}, \cite{LRE6}, \cite{LRE5}, Support Vector Machine (SVM) \cite{LRE1}, \cite{LRE2}, \cite{LRE6},~\cite{add1} and Multi-Layer Perceptron (MLP) \cite{LRE1}, \cite{LRE2} offer straightforward implementation and interpretable results for classification-based positioning. These algorithms are particularly suited for IPS applications due to their simple architectures and computational accessibility, making them practical for deployment without specialized hardware requirements. Data collection quality significantly impacts IPS accuracy, with improvements achieved through diverse device readings collected at various times \cite{LRE3}, Human Activity Recognition integration \cite{LRE4}, and coordination-based data collection applications \cite{LRE7}.

Deep learning approaches, such as Graph Neural Networks (GNNs) \cite{LRE2} and Convolutional Neural Networks (CNNs) \cite{LRE4}, have demonstrated superior performance over traditional ML algorithms when sufficient data points are available. Despite existing work, key environmental variables, such as the resolution of fingerprint grids and the spatial distribution of reference points, have not been comprehensively investigated. This paper emphasizes the critical role of these factors in IPS and systematically analyzes the most effective configurations for multi-floor environments. The objective is to enhance the accuracy, reliability, and overall performance of IPS by considering various influencing parameters and optimizing system design accordingly

%%%%%%%%%%%%%%%%%%%%%%%%%%%%%%%%%%%%%%%%%%%%%%%%%%%%%%%%%%%%%%%%%%%%%%%%%%%%%%%%%%%%%%%

\begin{figure}
	\includegraphics[width=\textwidth]{meth1.png}
	\caption{An overview of the research methodology. It divides the experiment into four main parts. (1) Identification of Environmental Factors, (2) Extraction of Environmental Factors, (3) Training of Fingerprint Matching Models, (4) Results and Evaluation}
	\label{fig:graph_step}
\end{figure}

\section{Research Methodology}
This paper adopts a systematic and empirical research methodology with the aim of identifying environmental factors that influence the IPS. This methodology is illustrated in Fig.~\ref{fig:graph_step}.

The methodology comprises four parts: (1) Environmental factor identification, (2) Data extraction, (3) Model training, and (4) Results analysis. Two datasets were created with varying grid sizes and reference points to train kNN, SVM, RF, MLP, XGBoost, and XGBoost-based RF models, selected for their classification effectiveness and accessibility.

\subsection{Identification of Environmental Factors}
The environment was mapped by taping floors into grids matching the digital floor plan. Each grid received a unique identifier across floors, assuming distinct BSSID signal distributions enable effective classification.

Initial data collection captured all detectable access points, including external sources from neighboring buildings. To prevent feature space complexity, filtering retained only campus network access points. As shown in Fig. \ref{fig:heatmap008}, low-strength signals may introduce training noise.

Two key factors emerged: grid size (balancing precision vs. data collection practicality) and low-relevance RSSI values requiring filtering for optimal model performance.

\begin{figure}[htbp]
	\centerline{\includegraphics[scale=0.15]{meth3.jpg}}
	\caption{A heatmap that shows different access point’s RSSI values (referred to as BSSID) for a grid.}
	\label{fig:heatmap008}
\end{figure}

\subsection{Extraction of Environmental Factors and Datasets}
The identification of the two factors in the previous section enables adjustments to find optimal settings for maximizing IPS performance. 

Traditional IPS evaluation relies on precise global coordinates, which are impractical to obtain indoors without professional surveying equipment. Even with such equipment, establishing accurate reference points requires extensive coordination with building management to modify structural elements for satellite signal reception, contradicting the goal of creating simple and accessible positioning systems. To address this limitation and enable practical IPS evaluation, two key metrics are defined for this study: Average Grid from Target (AGT) and Average Distance from Target (ADT).

\begin{equation}
	AGT = \sqrt{(\Delta X)^2 + (\Delta Y)^2}
	\label{eq:agt}
\end{equation}
$\Delta X$ and $\Delta Y$ represent horizontal and vertical axes, respectively. This formulation stems from treating each grid as a discrete unit within the map. Given that the grid layout inherently defines the number of divisions along the $X$ and $Y$ axes, offset can be directly computed as the distance between predicted and actual grid positions. Treating offsets as distance units, the measure naturally adapts to different grid sizes, providing a straightforward means to visualize performance and a standardized measure across different grid sizes.

\begin{equation}
	ADT = AGT \times \sqrt{{g_w}^2 + {g_h}^2}
	\label{eq:adt}
\end{equation}
$g_w$ and $g_h$ denote the grid width and height, respectively. While AGT captures classification error in grid labels, ADT introduces a precision-accuracy trade-off by scaling AGT with the grid's diagonal distance. These metrics offer significant advantages over traditional coordinate-based evaluation: AGT provides intuitive grid-based error measurement that adapts naturally to different grid sizes, while ADT enables direct comparison of positioning systems across varying spatial resolutions. Unlike GPS-based accuracy measurements that are impractical indoors, these metrics provide realistic performance assessment without requiring expensive surveying equipment or building modifications for reference point establishment.

\paragraph{Grid Size Requirements and Datasets} Grid size influences IPS performance. Its size influences the accuracy of the algorithm, as well as the ease of data collection. The following study \cite{LRE1} ran a model with grid sizes of 16.75 x 15m and reported high IPS performance in terms of AGT, as the larger the area, the more likely you’ll estimate the correct grid. The trade off was that the ADT was significantly lower, as there was a deviation from the actual point. This suggests that by minimising the grid size, we can achieve a higher precision. However the lower the precision, the harder the data collection process. 
Collecting data for varying grid sizes via the taping method is impractical, as repeatedly taping the area is not feasible for a small team. To address this, RSSI value interpolation was employed to generate larger grid sizes by aggregating smaller grids. The data collected from small grids (1m x 1m) is aggregated to construct larger grids by interpolating RSSI values from the combined smaller grids.

RSSI values were interpolated from 5 collection points per grid using standard averaging techniques.

\vspace{-5pt}
\begin{figure}[htbp]
	\centerline{\includegraphics[scale=0.5]{image14.jpg}}
	\caption{Visualizing grid resizing}
	\label{fig:vis_grid_resize}
\end{figure}
\vspace{-10pt}

\paragraph{Strategic BSSID Filtering as a Methodological Contribution} Previous IPS implementations \cite{LRE1} rely on comprehensive RSSI collection from all available access points, essentially employing a brute-force approach to feature acquisition. This study investigates whether comparable positioning accuracy can be achieved through selective feature filtering, addressing a key research question about the necessity of exhaustive signal collection.

Many access points exhibit low RSSI values as shown in fig. \ref{fig:heatmap008}, potentially introducing noise that could degrade model performance. Rather than blindly incorporating all available signals, this work implements a strategic filtering approach that selects only access points containing specific campus network identifiers. This methodology reduces feature dimensionality from 1799 to 378 BSSIDs while maintaining model effectiveness. To validate this approach, two datasets were created: one with all access points and one with filtered RSSI values. Comparing these datasets determines whether selective signal filtering can achieve similar performance to comprehensive collection methods.


\section{Result}
\vspace{-5pt}
\begin{table}[H]
	\centering
	\begin{minipage}{0.42\textwidth}
		\centering
		\caption{Total True Unique Grids by Floor}
		\label{tab:true_unique_grid}
		\vspace{2pt}
		\small
		\begin{tabular}{|l|c|c|} 
			\hline
			& \multicolumn{2}{c|}{\textbf{Floor}}  \\ 
			\hline
			\textbf{1x1 Grid Size} & 1st & 2nd \\ 
			\hline
			\textbf{Grid Count} & 309 & 174 \\
			\hline
		\end{tabular}
	\end{minipage}
	\hfill
	\begin{minipage}{0.52\textwidth}
		\centering
		\caption{Total BSSID Before and After Filtering}
		\label{tab:bssid_counts}
		\vspace{2pt}
		\small
		\begin{tabular}{|l|l|c|}
			\hline
			\multicolumn{2}{|l|}{\textbf{Collected Data Points}} & \textbf{\# BSSID} \\
			\hline
			\multirow{2}{*}{\textbf{Filtering}} & Non-Filtered & 1799 \\
			\cline{2-3}
			& Filtered & 378 \\
			\hline
		\end{tabular}
	\end{minipage}
\end{table}
\vspace{-20pt}
\begin{table}[H]
	\caption{Total Unique Grids for Each Grid Size Experiment}
	\label{tab:grid_size_variations}
	\vspace{2pt}
	\centering
	\small
	\begin{tabular}{|l|c|c|c|c|c|c|c|c|} 
		\hline
		\textbf{Grid Size} & 1x1 & 3x3 & 5x5 & 7x7 & 9x9 & 11x11 & 13x13 & 15x15 \\ 
		\hline
		\textbf{Total Grids} & 483 & 96 & 47 & 29 & 28 & 16 & 18 & 9 \\
		\hline
	\end{tabular}
\end{table}
\vspace{-15pt}

The dataset used in this study consisted of 12,640 data points collected across 1x1m\textsuperscript{2} grids distributed among the floors of the experimental setup. Filtering the available BSSID to utilize only SSID with specific identifiers resulted in 378 unique BSSID spanning across the 2 floors. The rationale behind filtering WiFi signals is to determine whether only easily identifiable BSSID can be utilized in implementing IPS. 9019 data points were collected on the 6th floor and 3621 data points on the 7th floor.

To explore how different parameter settings impact model training, various configurations were systematically tested, training the model multiple times under each setting. The table below highlights the best results observed. Notably, the highest accuracy and the best AGT do not come from the same parameter setting.

%\begin{equation}
%	AGT = \sqrt{(offset\_X)^2 + (offset\_Y)^2} 
%\end{equation}


This suggests that these metrics prioritize different aspects of performance—optimizing for accuracy does not necessarily yield the best AGT and vice versa. This insight provides a key perspective for analyzing the results in detail. 

Additionally, according to fig.~\ref{fig:AGT_dgrid_size}, from 7x7m grid size onwards, the best AGT measured across different models starts to plateau, indicating diminishing returns from increasing experiment grid size.

\begin{comment}
	\begin{figure}
		\centering
		\begin{subfigure}{0.4\textheight}
			\centerline{\includegraphics[scale=0.65]{image3.png}}
			\caption{Model Accuracy on Different grid size}
			\label{fig:acc_dgird_size}
		\end{subfigure}
		\hfill
		\begin{subfigure}{0.4\textheight}
			\centerline{\includegraphics[scale=0.65]{image1.png}}
			\caption{Model Average grid away from Target on Different grid size}
			\label{fig:AGT_dgrid_size}
		\end{subfigure}
	\end{figure}
\end{comment}

\begin{comment}
	\begin{figure}[htbp]
		\centering
		\includegraphics[scale=0.8]{image3.png}
		\caption{Model Accuracy on Different grid size}
		\label{fig:acc_dgird_size}
	\end{figure}
	
	\begin{figure}[htbp]
		\centering
		\includegraphics[scale=0.65]{image1.png}
		\caption{Model Average grid away from Target on Different grid size}
		\label{fig:AGT_dgrid_size}
	\end{figure}
\end{comment}

\vspace{-10pt}
\begin{figure}[hbt!]
	\begin{minipage}{0.45\textwidth}
		\centering
		\includegraphics[scale=0.5]{image3.png}
		\caption{Model Accuracy on Different grid size}
		\label{fig:acc_dgird_size}
	\end{minipage}
	\hfill
	\begin{minipage}{0.45\textwidth}
		\centering
		\includegraphics[scale=0.5]{image1.png}
		\caption{Model Average grid away from Target on Different grid size}
		\label{fig:AGT_dgrid_size}
	\end{minipage}
\end{figure}

\section{Discussion}
By conducting experiments across multiple grid sizes, the tradeoff between grid size and precision can be visualized. To quantify this, the average grid deviation from the target is calculated and multiplied by each grid size's diagonal length, as shown in fig. \ref{fig:AGT_dgrid_size}. This illustrates the expected deviation in predictions when a model is trained on a specific grid size, should an error occur.

While a 15x15m grid size performs well on the graph, it inherently limits accuracy to that resolution. In cases where predictions are correct, the location remains constrained within a 15×15m area, which may not be suitable for applications requiring higher precision—such as those needing to pinpoint areas smaller than this grid size.

Through this experiment, a 7x7m grid size was found to offer the best balance between accuracy and precision within the specific university learning center environment tested. The ADT demonstrates that utilizing a 7x7m grid size minimizes error while remaining small enough for precision-dependent applications in this particular setting. 

However, these findings may not generalize to all IPS implementations in different environments. The results suggest that increasing grid size does not necessarily improve accuracy or precision; in some cases, performance actually declines, as shown in the chart. This highlights the need for careful consideration when selecting a grid size based on specific IPS deployment requirements.


%\begin{equation}
%	ADT = AGT \times \sqrt{(grid\_width)^2 + (grid\_height)^2}.
%\end{equation}

\begin{figure}[htbp]
	\centerline{\includegraphics[scale=0.65]{image2.png}}
	\caption{Average Distance Error Across Different Grid Resolutions}
	\label{fig:Avg_dis_err}
\end{figure}

Training an IPS model involves significant computational challenges, particularly when dealing with high-dimensional feature spaces. The experiment implemented a simple Wi-Fi access point filtering approach, selecting only access points containing specific identifiers in their names. This drastically reduced the number of access points used as features, making the training process more efficient.

\begin{figure}[hbt!]
	\centering
	% First row
	\begin{minipage}{0.45\textwidth}
		\centering
		\includegraphics[width=\linewidth]{image5.png}
		\caption{RF Model accuracy with BSSID filtering (1x1)}
		\label{fig:rf_acc_filter}
	\end{minipage}
	\hfill
	\begin{minipage}{0.45\textwidth}
		\centering
		\includegraphics[width=\linewidth]{image6.png}
		\caption{RF Model accuracy without BSSID filtering (1x1)}
		\label{fig:rf_acc_nofilter}
	\end{minipage}
	
	\vspace{0.5cm} % vertical space between rows
	
	% Second row
	\begin{minipage}{0.45\textwidth}
		\centering
		\includegraphics[width=\linewidth]{image4.png}
		\caption{RF Model AGT with BSSID filtering (1x1)}
		\label{fig:rf_agt_filter}
	\end{minipage}
	\hfill
	\begin{minipage}{0.45\textwidth}
		\centering
		\includegraphics[width=\linewidth]{image7.png}
		\caption{RF Model AGT without BSSID filtering (1x1)}
		\label{fig:rf_agt_nofilter}
	\end{minipage}
\end{figure}

While the filtering method applied was relatively simple, it had a significant impact on computational feasibility. Observing fig.~\ref{fig:rf_acc_filter} and~\ref{fig:rf_acc_nofilter}, reducing the number of BSSID features from 1799 to 378 enabled model training across a wide range of grid sizes (1×1 to 15×15) efficiently. This reduction in feature dimensionality demonstrates the practical benefits of selective feature filtering for IPS implementations.

Fig.~\ref{fig:rf_agt_filter} and~\ref{fig:rf_agt_nofilter} suggest that filtering access points did not negatively impact the model's learning process. A comparison of training trajectories for the 1x1 grid experiment shows that the filtered model performed comparably to the unfiltered one, if not slightly better in terms of convergence. This suggests that reducing feature dimensionality not only accelerates training but may also make learning more stable.

From our observations we can conclude that AGT and grid size have an inverse relationship on all models. As grid-size increases the AGT decreases on every machine learning model. After creating models of 1x1 to 15x15 in increments of 2x2, we determined that 7x7m was the best model that balanced precision and accuracy for our test environment. This benefitted us as we were able to use these new measurements as empirical data upon which we could determine the best balance between grid-size and AGT. However; we must acknowledge that these results may not generalize to all environments due to different WiFi signal behaviour and building structures. Data filtration also aided in producing slightly better results with the BSSID filtered Model having an AGT of 15.16, and the non filtered Model having an AGT of 15.12. Filtering the dataset provided significant computational efficiency benefits, enabling practical model training across multiple grid sizes. This reduction in feature dimensionality not only accelerates training but may also contribute to more stable learning.





\section{Conclusion}
In conclusion, This paper extends prior work on classification-based IPS \cite{LRE1} by refining the implementation. Through our experiments across multiple grid sizes, we visualized the trade-offs between grid size and precision, showing that increasing grid size does not necessarily improve accuracy and may, in some cases, lead to worse performance. Our findings suggest that a 7×7m grid size offers the best balance between accuracy and precision within the tested university learning center environment, making it suitable for applications that require finer localization in similar indoor settings. These results are specific to the experimental environment and may not generalize to other building types, layouts, or WiFi infrastructure configurations due to variations in signal propagation characteristics and structural differences.

Additionally, we implemented a simple filtering method to limit the number of WiFi BSSID features, preventing excessive model complexity while maintaining comparable performance to an unfiltered approach. This method significantly reduced training time and computational requirements, allowing us to complete model training across multiple grid sizes efficiently. Our results indicate that reducing feature dimensionality not only accelerates training but may also contribute to more stable learning. To better assess IPS performance, we introduced two new evaluation metrics—Average Grid from Target (AGT) and Average Distance from Target (ADT)—which provide a clearer understanding of prediction deviation. Ultimately, our study highlights key considerations in IPS design, particularly in grid size selection and feature filtering, and offers insights for improving indoor positioning accuracy in practical implementations.

To summarize, we believe that the usage of Average Grid from Target (AGT)
and Average Distance from Target (ADT) will be beneficial as it will help people with mapping clearer prediction deviation and allow them to better pick a grid-size that matches their intended need. We also believe that filtering the data for our models provides many benefits, specifically allowing the model to be more computationally efficient as well as allowing our model to get better results. A potential idea for future work is the implementation of On-Device Prediction for our Model. This would allow for less-latency and improved responsiveness. To increase the model’s accuracy it is also possible to take this work into a different direction by switching from treating the problem as discrete to continuous localisation model.

\nocite{bgp4, add2, add3, add4}
\bibliographystyle{splncs04}
\bibliography{references}

\newpage
\appendix
\section{Implementation Details: Grid Alignment Challenges}

During the data collection phase, several practical challenges emerged that required careful consideration.

\begin{figure}[!htbp]
	\centering
	\includegraphics[width=\textwidth]{meth2.jpg}
	\caption{An example of taping that was done to create grids for data collection of RSSI values as fingerprints}
	\label{fig:taping}
\end{figure}

\subsection{Floor Tiling Variations}
Different layouts and tiling across areas presented challenges, as certain floors had distinct tiling with varying elevations and angles (as shown in Fig. \ref{fig:taping}). This was addressed by constructing grids based on previous reference points. Although minor measurement inaccuracies may exist, they are negligible for the purpose of this study.

\subsection{Edge Grid Effects}
While taping and collecting data points, another factor that could interfere with the IPS collection presented itself. The specific area inside the grid also affected how many RSSI values were present. For instance, collecting data in the top left corner of the grid could produce different RSSI values compared to the center of the grid. 

This issue could be addressed by either (1) removing the edge grids and (2) collecting fewer data points in the edge grid. However, both methods result in fewer data points being collected and have minimal impact on IPS performance. This is based on the assumption that, from a user experience perspective, being located at the edge of a grid represents a transitional state between adjacent grids. Meaning that, if misclassification occurs, it typically involves the neighboring grid and the correct grid itself. Therefore, for this study, the edge grid factor is negligible and omitted from further analysis.

\section{Hardware Implementation Considerations}
During the experimentation phase, specific hardware limitations significantly impacted the feasibility of training models with different feature sets. When attempting to train on the full, unfiltered BSSID set (1799 features), computational demands quickly became impractical—the RTX 3080 Ti GPU struggled even with the smallest grid size configurations. Although exact runtimes were not recorded, the difference in resource demands between filtered and unfiltered datasets was substantial.

By applying the simple filtering approach that reduced features from 1799 to 378 BSSIDs, model training became feasible across the entire range of grid sizes (1×1 to 15×15). This hardware constraint, while limiting, provided valuable insights into the practical utility of the full feature set and suggests that systematic feature selection or dimensionality reduction studies could be worthwhile in future works. The filtering approach not only made training computationally feasible but also demonstrated that selective feature reduction can maintain or even improve model performance while significantly reducing computational requirements.

\end{document}
