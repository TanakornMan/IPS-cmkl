\documentclass[10pt]{beamer}

\usetheme{metropolis}
\usepackage{appendixnumberbeamer}

\usepackage{booktabs}
\usepackage{graphicx}
\usepackage{pgfplots}
\usepgfplotslibrary{dateplot}

\usepackage{xspace}

\title{A Simplified Multi-Floor Classification-Based Indoor Positioning System Study}
\subtitle{Optimizing Grid Size and Feature Selection for ML-Based IPS}
\date{\today}
\author{Burin Intachuen\inst{1} \and Mhadhanagul Charoenphon\inst{1} \and Tanakorn Mankhetwit\inst{1} \and Charnon Pattiyanon\inst{2}}
\institute{
  \inst{1} Department of Computer Engineering, Mahidol University (International College), Thailand \\
  \inst{2} Department of AI and Computer Engineering, CMKL University, Thailand
}

\begin{document}

\maketitle

\begin{frame}{Table of contents}
  \setbeamertemplate{section in toc}[sections numbered]
  \tableofcontents[hideallsubsections]
\end{frame}

\section{Introduction}

\begin{frame}{Indoor Positioning Systems (IPS)}
  \begin{itemize}
    \item GPS and GNSS fail indoors due to signal attenuation
    \item Building materials (concrete, metal) block satellite signals
    \item IPS provides alternative solution for indoor navigation
    \item Applications: shopping malls, hospitals, university campuses
  \end{itemize}

  \begin{alertblock}{Challenge}
    How to optimize IPS performance in multi-floor environments?
  \end{alertblock}
\end{frame}

\begin{frame}{Current Approaches}
  \textbf{Traditional Methods:}
  \begin{itemize}
    \item Trilateration using BLE or Wi-Fi RSSI
    \item Fingerprinting-based positioning
    \item Limited by environmental noise
  \end{itemize}

  \vspace{1em}

  \textbf{Machine Learning Approaches:}
  \begin{itemize}
    \item kNN, Random Forest, SVM, Neural Networks
    \item Treat positioning as classification problem
    \item Better handling of noisy RSSI data
  \end{itemize}
\end{frame}

\begin{frame}{Research Motivation}
  \textbf{Previous work limitations:}
  \begin{itemize}
    \item Large grid sizes (16.75×15 m) used for simplicity
    \item Limited understanding of grid size impact
    \item No systematic study of feature filtering effects
  \end{itemize}

  \vspace{1em}

  \textbf{Our contributions:}
  \begin{enumerate}
    \item New evaluation metrics: AGT and ADT
    \item Grid size vs. precision trade-off analysis
    \item Feature filtering impact on model performance
    \item Practical insights for IPS implementation
  \end{enumerate}
\end{frame}

\section{Research Methodology}

\begin{frame}{Research Overview}
  \begin{figure}
    \centering
    \includegraphics[width=0.9\linewidth]{../figures/meth1.png}
    \caption{Four-part research methodology}
  \end{figure}
\end{frame}

\begin{frame}{Environmental Factors}
  \textbf{Key factors identified:}
  \begin{enumerate}
    \item \textbf{Grid Size}
    \begin{itemize}
      \item Larger grids $\rightarrow$ easier classification, lower precision
      \item Smaller grids $\rightarrow$ higher precision, more data collection effort
    \end{itemize}

    \vspace{0.5em}

    \item \textbf{Low-Relevance BSSIDs}
    \begin{itemize}
      \item Signals from distant access points
      \item Introduce noise and increase computational cost
      \item Need systematic filtering approach
    \end{itemize}
  \end{enumerate}
\end{frame}

\begin{frame}{RSSI Heatmap Analysis}
  \begin{figure}
    \centering
    \includegraphics[width=0.7\linewidth]{../figures/meth3.jpg}
    \caption{Heatmap showing RSSI values from different access points}
  \end{figure}
  \begin{itemize}
    \item Darker colors = lower signal strength
    \item Many low-relevance BSSIDs detected
  \end{itemize}
\end{frame}

\begin{frame}{New Evaluation Metrics}
  \textbf{Average Grid from Target (AGT):}
  \begin{equation*}
    AGT = \sqrt{(x_{target} - x_{estimate})^2 + (y_{target} - y_{estimate})^2}
  \end{equation*}

  \vspace{1em}

  \textbf{Average Distance from Target (ADT):}
  \begin{equation*}
    ADT = \sqrt{[w_{g} \times (x_{target} - x_{estimate})]^2 + [h_{g} \times (y_{target} - y_{estimate})]^2}
  \end{equation*}

  \begin{itemize}
    \item AGT: Grid-based distance (adaptable to any grid size)
    \item ADT: Physical distance in meters
  \end{itemize}
\end{frame}

\begin{frame}{Grid Interpolation Approach}
  \begin{columns}[T,onlytextwidth]
    \column{0.5\textwidth}
      \begin{figure}
        \includegraphics[width=\linewidth]{../figures/image14.jpg}
        \caption{Grid aggregation visualization}
      \end{figure}

    \column{0.5\textwidth}
      \textbf{Benefits:}
      \begin{itemize}
        \item Avoid re-taping entire area
        \item Generate multiple grid sizes from 1×1m base
        \item Reduce human measurement errors
      \end{itemize}

      \vspace{1em}

      \textbf{Method:}
      \begin{itemize}
        \item Average RSSI from 5 sampling points
        \item Test 8 grid sizes: 1×1m to 15×15m
      \end{itemize}
  \end{columns}
\end{frame}

\begin{frame}{Dataset Overview}
  \begin{table}
    \caption{Dataset metadata by grid size}
    \small
    \begin{tabular}{@{} ccccc @{}}
      \toprule
      \textbf{Floor} & \textbf{Grid Size} & \textbf{\# Grids} & \textbf{\# Data Points} \\
      \midrule
      1st & 1×1m & 309 & 8,086 \\
      1st & 7×7m & 17 & 445 \\
      1st & 15×15m & 5 & 130 \\
      \midrule
      2nd & 1×1m & 174 & 4,554 \\
      2nd & 7×7m & 12 & 314 \\
      2nd & 15×15m & 4 & 105 \\
      \bottomrule
    \end{tabular}
  \end{table}

  \begin{itemize}
    \item Total: 12,640 RSSI samples
    \item Filtered: 378 BSSIDs (from 1,799)
  \end{itemize}
\end{frame}

\begin{frame}{Machine Learning Models}
  \textbf{Models evaluated:}
  \begin{itemize}
    \item k-Nearest Neighbor (kNN)
    \item Random Forest (RF)
    \item Support Vector Machine (SVM)
    \item Multi-Layer Perceptron (MLP)
    \item XGBoost
    \item XGBoost + Random Forest hybrid
  \end{itemize}

  \vspace{1em}

  \textbf{Training approach:}
  \begin{itemize}
    \item Standard process with hyperparameter tuning
    \item Consistent dataset across all models
    \item Comparison of filtered vs. unfiltered BSSIDs
  \end{itemize}
\end{frame}

\section{Experiments and Results}

\begin{frame}{Accuracy by Grid Size}
  \begin{figure}
    \centering
    \includegraphics[width=0.7\linewidth]{../figures/overview_filtered_accuracy.png}
    \caption{Model accuracy across different grid sizes (Filtered)}
  \end{figure}

  \begin{itemize}
    \item Larger grids $\rightarrow$ higher accuracy
    \item 7×7m grid achieves ~80\% accuracy
    \item RF and XGBoost perform best
  \end{itemize}
\end{frame}

\begin{frame}{AGT Results}
  \begin{figure}
    \centering
    \includegraphics[width=0.7\linewidth]{../figures/overview_filtered_agt.png}
    \caption{Average Grid from Target (Filtered)}
  \end{figure}

  \begin{itemize}
    \item AGT stabilizes at 7×7m grid size
    \item Smaller grids show higher positioning errors
  \end{itemize}
\end{frame}

\begin{frame}{ADT Results}
  \begin{figure}
    \centering
    \includegraphics[width=0.7\linewidth]{../figures/overview_filtered_adt.png}
    \caption{Average Distance from Target in meters (Filtered)}
  \end{figure}

  \begin{itemize}
    \item 7×7m grid minimizes physical distance error
    \item Optimal balance for precision-dependent applications
  \end{itemize}
\end{frame}

\begin{frame}{Feature Filtering Impact}
  \begin{columns}[T,onlytextwidth]
    \column{0.5\textwidth}
      \begin{figure}
        \includegraphics[width=\linewidth]{../figures/overview_filtered_accuracy.png}
        \caption{Filtered (378 BSSIDs)}
      \end{figure}

    \column{0.5\textwidth}
      \begin{figure}
        \includegraphics[width=\linewidth]{../figures/overview_unfiltered_accuracy.png}
        \caption{Unfiltered (1,799 BSSIDs)}
      \end{figure}
  \end{columns}

  \vspace{0.5em}

  \textbf{Filtering benefits:}
  \begin{itemize}
    \item Comparable accuracy with 79\% fewer features
    \item More stable performance
    \item Reduced computational requirements
  \end{itemize}
\end{frame}

\section{Discussion}

\begin{frame}{Key Findings}
  \begin{enumerate}
    \item \textbf{Optimal Grid Size: 7×7m}
    \begin{itemize}
      \item Best balance between accuracy and spatial resolution
      \item ~80\% accuracy with lowest AGT/ADT values
    \end{itemize}

    \vspace{0.5em}

    \item \textbf{Feature Filtering Works}
    \begin{itemize}
      \item 79\% reduction in features (1,799 $\rightarrow$ 378)
      \item Maintained performance, improved stability
      \item Significantly reduced training time
    \end{itemize}

    \vspace{0.5em}

    \item \textbf{Best Models: RF and XGBoost}
    \begin{itemize}
      \item Consistently outperformed other models
      \item Well-suited for fingerprint-matching tasks
      \item Better than complex deep learning in this setting
    \end{itemize}
  \end{enumerate}
\end{frame}

\begin{frame}{Practical Implications}
  \textbf{For IPS Implementation:}
  \begin{itemize}
    \item Don't assume smaller grids are always better
    \item Moderate grid sizes can be optimal
    \item Simple feature filtering is effective
    \item Traditional ML can outperform deep learning
  \end{itemize}

  \vspace{1em}

  \textbf{Limitations:}
  \begin{itemize}
    \item Site-specific results (university campus)
    \item May need recalibration for different environments
    \item Architecture and materials affect performance
  \end{itemize}
\end{frame}

\section{Conclusion}

\begin{frame}{Summary}
  \textbf{What we achieved:}
  \begin{itemize}
    \item Identified optimal 7×7m grid size for multi-floor IPS
    \item Introduced AGT and ADT metrics for standardized evaluation
    \item Demonstrated effective BSSID filtering approach
    \item Showed RF and XGBoost superiority
  \end{itemize}

  \vspace{1em}

  \textbf{Future directions:}
  \begin{itemize}
    \item Advanced deep learning models (attention-based, GNN-MLP)
    \item Real-time adaptive grid configurations
    \item Synthetic data augmentation and sensor fusion
    \item Transfer learning across buildings
  \end{itemize}
\end{frame}

\begin{frame}[standout]
  Questions?
\end{frame}

\appendix

\begin{frame}{Implementation Challenges}
  \textbf{Floor tiling variations:}
  \begin{itemize}
    \item Different elevations and angles
    \item Addressed by reference-point-based grid construction
  \end{itemize}

  \vspace{1em}

  \textbf{Edge grid effects:}
  \begin{itemize}
    \item Corner vs. center RSSI variations
    \item Minimal impact on overall performance
    \item Misclassifications typically involve adjacent grids
  \end{itemize}

  \vspace{1em}

  \textbf{Hardware constraints:}
  \begin{itemize}
    \item RTX 3080 Ti struggled with unfiltered dataset
    \item Filtering made training feasible across all grid sizes
  \end{itemize}
\end{frame}

\end{document}
